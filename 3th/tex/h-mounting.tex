\section{Practical Mounting of the Antenna}

For the HF vertical dipole antenna, a reasonable and practical method of mounting
the antenna involves the use of guy wires. This is done by having a trio of
wires evenly spaced radially about the tower connecting to bases prevening wind from causing to
buckle. On taller towers, there are several segments of guy wires. Following
advice from the ARRL, guy wire segments should not be seperated more than 35
feet (10.67 meters), and all guy wires should connect to a common base roughly
60\%-80\% of the tower length out from the base of the antenna. \cite{antbook}

Therefore, for a $\frac{\lambda}{2}$ dipole antenna for $\lambda=40$m, there
should be a total of at least two guy wire segments; the first segment
connecting at 10.67m with a length of 21.3 meters to a base firmly anchored in
the ground roughly 14m out. The second set should be at or near the top of the antenna at 20m
and have a length near 24m.
